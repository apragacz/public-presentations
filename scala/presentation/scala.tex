\documentclass{beamer}
\usetheme{default}
\usepackage{listings}
\usepackage{color}
\usepackage{hyperref}

\definecolor{dkgreen}{rgb}{0,0.6,0}
\definecolor{gray}{rgb}{0.5,0.5,0.5}
\definecolor{mauve}{rgb}{0.58,0,0.82}

\def \lstscalaset {\lstset{ language=Java, basicstyle=\scriptsize,frame=single,
    keywordstyle=\color{blue},stringstyle=\color{mauve},commentstyle=\color{dkgreen}}}

\def \lstjavaset {\lstset{ language=Java, basicstyle=\scriptsize,frame=single,
    keywordstyle=\color{blue},stringstyle=\color{mauve},commentstyle=\color{dkgreen}}}


\title{Scala}
\subtitle{Cool features \& WATs}
\author{Andrzej Pragacz}

\begin{document}


\titlepage



\section{Basic syntax}

\begin{frame}{Scala}
Scala...
\begin{itemize}
\item is statically typed
\item clererly mixes object oriented and functional paradigms
\item is a JVM and .NET language
\end{itemize}
\end{frame}

\begin{frame}{Example - Hello World}
\lstscalaset
\lstinputlisting{scala/basic/hello.scala}

And thats it! Compare it to the Java version:

\lstjavaset
\lstinputlisting{scala/basic/hello.java}

\end{frame}

\begin{frame}{Type inference}
\begin{itemize}
\item scala can infer the types by specifing just the types of the parameters of the function
\end{itemize}
\lstscalaset
\lstinputlisting{scala/basic/inference_weightedavg.scala}
\end{frame}

\begin{frame}{Conditionals}
\begin{itemize}
\item conditionals are basically in the form if (condition) valueIfTrue else valueIfFalse
\end{itemize}

\lstscalaset
\lstinputlisting{scala/basic/conditionals_abs.scala}

\end{frame}

\begin{frame}{Blocks}
\begin{itemize}
\item block can contain multiple expressions
\item expressions can be separated by either a newline or a semicolon
\item the block is evaluated to the last expression in it
\item definitions in the block are contained in it
\item you can access the ``parent'' definitions  (global definitions / definitions from upper blocks in given block)
\item definitions in the block shadow the ``parent'' definitions
\end{itemize}

\lstscalaset
\lstinputlisting{scala/basic/block_multiply.scala}

\end{frame}

\begin{frame}{Type inference - recurrent function definitions}
\begin{itemize}
\item in addition to the types of input parameters, recursive functions need
also the return type to be defined
\end{itemize}

\lstscalaset
\lstinputlisting{scala/basic/rec_factorial.scala}
\end{frame}


\begin{frame}{Basic data structure - list}
\lstscalaset
\lstinputlisting{scala/basic/list_reverse.scala}
\end{frame}

\section{Cool features}

\begin{frame}{Anonymous functions \& higher order functions}
\begin{itemize}
\item In scala, you can construct anonymous functions (``lambdas'')
which allow you to put code in place, without defining additional functions somewhere
\item also most collections (e.g. lists) implement methods which are high order
functions which can
take some function and return something which also can be a function
\end{itemize}
\end{frame}

\begin{frame}{Example - sum of squares}
\lstscalaset
\lstinputlisting{scala/cool/high_order_sum.scala}
\end{frame}

\begin{frame}{val, var, def and lazy val}
\begin{itemize}
\item in scala you have different way to define your primitives
\item val defines something which is evaluated when it is defined
\item var is the same, except it can be reassigned
\item def is evaluated each time when it is accessed
\item lazy val is evaluated only the first time when it is accessed
\end{itemize}
\end{frame}

\begin{frame}{example}
\end{frame}

\begin{frame}{lazy function argument evaluation}
\begin{itemize}
\item in most languages the parameters of a function are evaluated, and
then passed to the function
\item scala allows you to control when the expression supplied as function
parameter should be evaluated
\end{itemize}
\end{frame}

\begin{frame}{Example - non-eager AND operator}
\lstscalaset
\lstinputlisting{scala/cool/noneager_and.scala}
\end{frame}

\begin{frame}{Example - non-eager while}
\lstscalaset
\lstinputlisting{scala/cool/noneager_while.scala}
\end{frame}

\begin{frame}{Tail recursion}
\lstscalaset
\lstinputlisting{scala/cool/tailrec_factorial.scala}
\end{frame}

\begin{frame}{Defining operators}
\begin{itemize}
\item in scala you do not overload operators
\item you define them
\item the operator precedence is based on first character of operator:
    \begin{enumerate}
    \item (all letters)
    \item $|$
    \item \^{}
    \item \&
    \item $<$ $>$
    \item = !
    \item :
    \item + -
    \item * / \%
    \item(all other special characters)

    \end{enumerate}
\end{itemize}
\end{frame}

\begin{frame}{Example - fraction class}
\lstscalaset
\lstinputlisting{scala/cool/operators_fractions1.scala}
\end{frame}

\begin{frame}{Example - fraction class (cont.)}
\lstscalaset
\lstinputlisting{scala/cool/operators_fractions2.scala}
\end{frame}

\begin{frame}{Callable objects}
\begin{itemize}
\item In scala, you can make the objects callable by adding method named
``apply'' which takes required parameters
\end{itemize}
\end{frame}

\begin{frame}{Example - two argument operators}
\lstscalaset
\lstinputlisting{scala/cool/callable_operators.scala}
\end{frame}

\begin{frame}{Case classes and pattern matching}
\begin{itemize}
\item scala allows to define special classes, which can be ``decomposable''
in some way
\item it may be useful when the algorithm needs to
\end{itemize}
\end{frame}

\begin{frame}{Example - expression evaluation}
\lstscalaset
\lstinputlisting{scala/cool/patmat_expressions.scala}
\end{frame}

\begin{frame}{Type system}
\begin{itemize}
\item the type Any is the base type
\item the type AnyVal is the base type of classes like Int, Double, Boolean
\item the type AnyRef is the base type of ScalaObject class, which is base type for instance for List
\item the String type is not a subtype of ScalaObject, but actualy the Java String
\item null has special Null type, which is a subtype of all AnyRef subtypes
\item there is also a special type Noting, which is subtype of all types
\item what is the type of empty list?
\end{itemize}
\end{frame}

\begin{frame}{Type Co(ntra)variance}
\begin{itemize}
\item if $A <: B$,  then we say that $A$ is a subtype of $B$
\item in this case, if  $C[A] <: C[B]$ then $C$ is covariant
\item in this case, if  $C[B] <: C[A]$ then $C$ is contravariant
\item If $A <: B$, then everything one can to do with a value of
type $B$ one should also be able to do with a value of type $A$.
\item Covariance is designated by $+$, contravariance by $-$
\end{itemize}
\end{frame}

\begin{frame}{Example - Function type \& List type}
\lstscalaset
\lstinputlisting{scala/cool/variance_functions.scala}

\lstscalaset
\lstinputlisting{scala/cool/variance_lists.scala}

And what about arrays?
\end{frame}


\begin{frame}{List comprehensions}
\begin{itemize}
\item using functions like map, reduce, filter, you can easily find in the collections
what you are looking for
\item unfortunately, this does not always looks clear
\item fortunately, scala has list comprehensions
\end{itemize}
\end{frame}

\begin{frame}{Examples - String reverse \& dot product}
\lstscalaset
\lstinputlisting{scala/cool/listcomp_dot_product.scala}

\lstscalaset
\lstinputlisting{scala/cool/listcomp_reverse_string.scala}
\end{frame}

\begin{frame}{Scala REPL}
\end{frame}

\begin{frame}{SBT}
\end{frame}

\begin{frame}{Tuples}
\begin{itemize}
\item scala allows to create tuples, which basically are structures and may hold
objects of different types
\item it may be questionable whether tuples should be used as they lack semantical
information, but still they can be handy sometimes
\end{itemize}
\end{frame}

\begin{frame}{Example - returning multiple values}
\lstscalaset
\lstinputlisting{scala/cool/tuples_add_and_mul.scala}
\end{frame}

\begin{frame}{streams}
\begin{itemize}
\item streams are basically lazy evaluated lists
\item they can be potentially infinite
\end{itemize}
\end{frame}

\begin{frame}{Example - prime stream}
\lstscalaset
\lstinputlisting{scala/cool/streams_primes.scala}
\end{frame}

\begin{frame}{Actors}
\end{frame}

\section{WATs}

\begin{frame}{Array index accessors}
\begin{itemize}
\item arrays are accessed like they would be functions - $a(0)$
\end{itemize}
\end{frame}

\begin{frame}{Partial application}
\lstscalaset
\lstinputlisting{scala/wat/partial_application.scala}
\end{frame}

\begin{frame}{Partial application continued}
\lstscalaset
\lstinputlisting{scala/wat/partial_application2.scala}
\end{frame}

\begin{frame}{Tuple accessors}
\lstscalaset
\lstinputlisting{scala/wat/tuple_accessors.scala}
\end{frame}

\end{document}
