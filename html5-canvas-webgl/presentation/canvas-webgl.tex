\documentclass{beamer}
\usetheme{default}
\begin{document}



\section{Canvas}

\begin{frame}{Brief history of canvas element}
\end{frame}

\begin{frame}{What can you do with canvas?}
\begin{itemize}
    \item draw straight lines, curves, circles, rects
    \item draw objects using paths
    \item draw images
    \item fill objects with gradients
    \item draw DOM objects
    \item transform objects
    \item save/restore canvas state
    \item compose (blend) objects in various ways
\end{itemize}
\end{frame}

\begin{frame}{Tips}
\begin{itemize}
    \item Pre-render similar primitives or repeating objects on an off-screen canvas.
    \item Batch canvas calls together (for example, draw a poly-line instead of multiple separate lines).
    \item Avoid floating point coordinates and use integers instead.
    \item Avoid unnecessary canvas state changes.
    \item Render screen differences only, not the whole new state.
    \item Use multiple layered canvases for complex scenes.
    \item Avoid shadowBlur.
    \item With animations, use requestAnimationFrame.
    \item Test performance with JSPerf.

\end{itemize}
\end{frame}


\begin{frame}{Libraries}

libCanvas is powerful and lightweight canvas framework

Processing.js is a port of the Processing visualization language

EaselJS is a library with a Flash-like API

PlotKit is a charting and graphing library

Rekapi is an animation keyframing API for Canvas

PhiloGL is a WebGL framework for data visualization, creative coding and game development.

JavaScript InfoVis Toolkit creates interactive 2D Canvas data visualizations for the Web.


\end{frame}



\section{WebGL}

\end{document}
