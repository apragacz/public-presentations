\documentclass{beamer}
\usetheme{default}
\usepackage{listings}
\usepackage{color}
\usepackage{hyperref}

\definecolor{dkgreen}{rgb}{0,0.6,0}
\definecolor{gray}{rgb}{0.5,0.5,0.5}
\definecolor{mauve}{rgb}{0.58,0,0.82}

\title{Canvas and WebGL}
\author{Andrzej Pragacz}

\begin{document}


\titlepage

\section{Canvas}

\begin{frame}{Canvas}
Basically a new HTML element, introduced in HTML5, which allows to draw objects on itself.
\end{frame}

\begin{frame}{Support}
\begin{itemize}
\item Chrome - 9.0+
\item Firefox - 2.0+
\item Opera - 9.0+
\item Safari - 3.1+
\item Safari (mobile) - 3.2+
\item Android Browser - 2.1+
\item Internet Explorer - 9.0+
\end{itemize}
\end{frame}

\begin{frame}{What can you do with canvas?}
\begin{itemize}
    \item draw straight lines, curves, circles, rects
    \item draw objects using paths
    \item draw images
    \item fill objects with gradients
    \item draw DOM objects
    \item transform objects
    \item save/restore canvas state
    \item compose (blend) objects in various ways
\end{itemize}
\end{frame}


\begin{frame}{Getting context}

assuming we have somewhere defined in the HTML code:

\lstset{ language=HTML, basicstyle=\scriptsize,frame=single}
\lstinputlisting{canvas/tag.html}

we get the context:

\lstset{ language=Java, basicstyle=\scriptsize,frame=single,
    keywordstyle=\color{blue},stringstyle=\color{mauve},commentstyle=\color{dkgreen}}
\lstinputlisting{canvas/getcontext.js}

And we're ready!

\end{frame}


\begin{frame}{Drawing background rectangle}

black:

\lstset{ language=Java, basicstyle=\scriptsize,frame=single,
    keywordstyle=\color{blue},stringstyle=\color{mauve},commentstyle=\color{dkgreen}}
\lstinputlisting{canvas/draw_rect.js}

with gradient:

\lstset{ language=Java, basicstyle=\scriptsize,frame=single,
    keywordstyle=\color{blue},stringstyle=\color{mauve},commentstyle=\color{dkgreen}}
\lstinputlisting{canvas/draw_gradient.js}

also, radial gradient can be used

\end{frame}


\begin{frame}{Drawing pentagon filled with semi-transparent color and shadow}

\lstset{ language=Java, basicstyle=\scriptsize,frame=single,
    keywordstyle=\color{blue},stringstyle=\color{mauve},commentstyle=\color{dkgreen}}
\lstinputlisting{canvas/draw_ngon.js}

\end{frame}


\begin{frame}{Drawing pentagon boundary}

\lstset{ language=Java, basicstyle=\scriptsize,frame=single,
    keywordstyle=\color{blue},stringstyle=\color{mauve},commentstyle=\color{dkgreen}}
\lstinputlisting{canvas/draw_ngon_stroked.js}

\end{frame}


\begin{frame}{Transformations}

the state of context can be saved (with all properties, like fillColor, globalAlpha etc.)
this is useful when we change the coordinates system

\lstset{ language=Java, basicstyle=\scriptsize,frame=single,
    keywordstyle=\color{blue},stringstyle=\color{mauve},commentstyle=\color{dkgreen}}
\lstinputlisting{canvas/transform.js}

and then restored back:

\lstset{ language=Java, basicstyle=\scriptsize,frame=single,
    keywordstyle=\color{blue},stringstyle=\color{mauve},commentstyle=\color{dkgreen}}
\lstinputlisting{canvas/restore.js}


\end{frame}


\begin{frame}{Drawing images}

\lstset{ language=Java, basicstyle=\scriptsize,frame=single,
    keywordstyle=\color{blue},stringstyle=\color{mauve},commentstyle=\color{dkgreen}}
\lstinputlisting{canvas/draw_image.js}

canvas objects can also be drawn!

\end{frame}



\begin{frame}{Tips}
\begin{itemize}
    \item Pre-render similar primitives or repeating objects on an off-screen canvas.
    \item Batch canvas calls together (for example, draw a poly-line instead of multiple separate lines).
    \item Avoid floating point coordinates and use integers instead.
    \item Avoid unnecessary canvas state changes.
    \item Render screen differences only, not the whole new state.
    \item Use multiple layered canvases for complex scenes.
    \item Avoid shadowBlur.
    \item With animations, use requestAnimationFrame.
    \item Test performance with JSPerf.

\end{itemize}
\end{frame}


\section{WebGL}

\begin{frame}{WebGL}

\begin{itemize}
\item is based on OpenGL ES 2.0, which is based on OpenGL 2.0
\item provides an API for 3D graphics
\item uses the HTML5 canvas element
\end{itemize}

\end{frame}


\begin{frame}{Support}

\begin{itemize}
\item Availability depends on graphics card driver (with opengl 2.0 support) and
support from the browser (google chrome, firefox)
\item Security issues - Cross-Origin Resource Sharing
\item Mozilla Firefox - 4.0+ (CORS - 8.0+)
\item Google Chrome - 9+ (CORS - 13.0+)
\item Safari 5.1+ (disabled by default)
\item Opera - 11+ (disabled by default)
\item Internet Explorer - nope (but there are plugins, like The Chrome Frame and IEWebGL)

\item Sometimes the support from graphics driver (if any) is not enough - on
linux software emulation like Mesa library can be helpful
\end{itemize}

\end{frame}

\begin{frame}{WebGL philosophy}

\begin{itemize}
\item instead of glFunctionDoingSomething() we have gl.functionDoingSomething()
\item instead of glBegin(), glVertex*() calls we rather use vertex buffers and
let the shaders do all the work
\end{itemize}

\end{frame}


\begin{frame}{Initializing}

\lstset{ language=Java, basicstyle=\scriptsize,frame=single,
    keywordstyle=\color{blue},stringstyle=\color{mauve},commentstyle=\color{dkgreen}}
\lstinputlisting{webgl/init.js}

... but that's not all!

\end{frame}


\begin{frame}{Initializing cont.}

\lstset{ language=Java, basicstyle=\scriptsize,frame=single,
    keywordstyle=\color{blue},stringstyle=\color{mauve},commentstyle=\color{dkgreen}}
\lstinputlisting{webgl/init_ext.js}

\end{frame}


\begin{frame}{Shaders}

vertex shader (GLSL)
\lstset{ language=C, basicstyle=\scriptsize,frame=single,
    keywordstyle=\color{blue},stringstyle=\color{mauve},commentstyle=\color{dkgreen}}
\lstinputlisting{webgl/vertex_shader.txt}

fragment shader (GLSL)
\lstset{ language=C, basicstyle=\scriptsize,frame=single,
    keywordstyle=\color{blue},stringstyle=\color{mauve},commentstyle=\color{dkgreen}}
\lstinputlisting{webgl/fragment_shader.txt}

\end{frame}


\begin{frame}{Loading shaders}

\lstset{ language=Java, basicstyle=\scriptsize,frame=single,
    keywordstyle=\color{blue},stringstyle=\color{mauve},commentstyle=\color{dkgreen}}
\lstinputlisting{webgl/load_vertex_shader.js}

\lstset{ language=Java, basicstyle=\scriptsize,frame=single,
    keywordstyle=\color{blue},stringstyle=\color{mauve},commentstyle=\color{dkgreen}}
\lstinputlisting{webgl/load_fragment_shader.js}

\end{frame}


\begin{frame}{Shader program}

\lstset{ language=Java, basicstyle=\scriptsize,frame=single,
    keywordstyle=\color{blue},stringstyle=\color{mauve},commentstyle=\color{dkgreen}}
\lstinputlisting{webgl/shader_prog.js}

\end{frame}


\begin{frame}{Creating vertex buffer}

\lstset{ language=Java, basicstyle=\scriptsize,frame=single,
    keywordstyle=\color{blue},stringstyle=\color{mauve},commentstyle=\color{dkgreen}}
\lstinputlisting{webgl/buffer.js}

\end{frame}


\begin{frame}{Finally, drawing the scene}

\lstset{ language=Java, basicstyle=\scriptsize,frame=single,
    keywordstyle=\color{blue},stringstyle=\color{mauve},commentstyle=\color{dkgreen}}
\lstinputlisting{webgl/draw.js}

\end{frame}


\begin{frame}{More advanced Shaders (with texturing)}

vertex shader (GLSL)
\lstset{ language=C, basicstyle=\scriptsize,frame=single,
    keywordstyle=\color{blue},stringstyle=\color{mauve},commentstyle=\color{dkgreen}}
\lstinputlisting{webgl/tex_vertex_shader.txt}

\end{frame}


\begin{frame}{More advanced Shaders (with texturing)}

fragment shader (GLSL)
\lstset{ language=C, basicstyle=\scriptsize,frame=single,
    keywordstyle=\color{blue},stringstyle=\color{mauve},commentstyle=\color{dkgreen}}
\lstinputlisting{webgl/tex_fragment_shader.txt}

\end{frame}

\begin{frame}{Tutorials}
\begin{itemize}
\item \url{https://developer.mozilla.org/en-US/docs/HTML/Canvas}
\item \url{http://designconcept.webdev20.pl/articles/html5-canvas-podstawy/}
\item \url{http://learningwebgl.com/blog/}
\item \url{http://www.khronos.org/webgl/}
\item \url{http://blog.tojicode.com/}
\end{itemize}

\end{frame}

\begin{frame}{Demos}
\begin{itemize}
\item \url{https://developer.mozilla.org/samples/raycaster/RayCaster.html}
\item \url{http://www.ibiblio.org/e-notes/webgl/webgl.htm}
\item \url{http://media.tojicode.com/q3bsp/}
\item \url{https://developer.mozilla.org/en-US/demos/detail/bananabread}
\end{itemize}

\end{frame}

\begin{frame}{Libraries/Frameworks (canvas)}

\begin{itemize}
\item KinecticJS
\item libCanvas is powerful and lightweight canvas framework
\item Processing.js is a port of the Processing visualization language
\item EaselJS is a library with a Flash-like API
\item PlotKit is a charting and graphing library
\item Rekapi is an animation keyframing API for Canvas
\item PhiloGL is a WebGL framework for data visualization, creative coding and game development.
\item JavaScript InfoVis Toolkit creates interactive 2D Canvas data visualizations for the Web.
\end{itemize}


\end{frame}


\end{document}
